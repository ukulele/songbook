%% The Balham Ukulele Society Songbook Project
%% Basic Songbook
%% Master A4 version
%% 
%% Copyright (C) 2013 Balham Ukulele Society
%% Maintainer Vish Vishvanath <ukulele@tfto.com>--author-maintained. 
%% 
%% The Balham Ukulele Society Songbooks project by Vish Vishvanath, Matt Gunning, Alyn Gwyndaf, Charlie Ullman, et al
%% is licensed under a Creative Commons Attribution-NonCommercial-ShareAlike 3.0 Unported License.
%%
%% http://creativecommons.org/licenses/by-nc-sa/3.0/deed.en_US
%%
\def\fileversion{1.0} \def\filedate{12/02/2013} 

\documentclass[11pt,a4paper,oneside]{book}

% PDF Metadata and Links
\usepackage[pdfauthor={Vish Vishvanath et al},
            pdftitle={The Balham Ukulele Society Songbook Project},
            pdfsubject={Basic Songbook},
            pdfkeywords={Ukulele, Balham, London},
            pdfproducer={Latex},
            pdfcreator={pdflatex via Emacs, Vim, Textmate}
            colorlinks=true
            dfstartview=FitV
            linkcolor=blue
            citecolor=blue
            urlcolor=blue]{hyperref}


% Beware of modifying this - you could easily break the layout.
\usepackage[top=1.5cm, bottom=1in, left=1in, right=2in, textwidth=15cm, textheight=26cm, marginparwidth=0in]{geometry}

% Set up the fonts if you know what you're doing
% \usepackage{fontspec,xltxtra,xunicode}
% \defaultfontfeatures{Mapping=tex-text} % converts LaTeX specials (``quotes'' --- dashes etc.) to unicode
% \setromanfont [Scale=0.8, Ligatures={Common}, Numbers={OldStyle}]{Minion Pro}
\renewcommand{\rmdefault}{ppl} % rm
\linespread{1.05}        % Palatino needs more leading
\usepackage[scaled]{helvet} % ss
\usepackage{courier} % tt
\usepackage{euler} % math
%\usepackage{eulervm} % a better implementation of the euler package (not in gwTeX)
\normalfont
\usepackage[T1]{fontenc}

\usepackage[parfill]{parskip}
\usepackage{amssymb}
% \usepackage{lmodern}

% Mini tables of content
 \usepackage{minitoc}

% add some colour
\usepackage[usenames,dvipsnames]{xcolor}

% need linenumbers so we can keep track of the song while playing in the dimly-lit pub
\usepackage[pagewise]{lineno}
\renewcommand\linenumberfont{\color{SkyBlue}\small}

% distance between margin and body
\setlength{\marginparsep}{1in}

% chords
\usepackage{gchords}

% Ukulele Settings
\renewcommand\strings{4}          % number of strings on your guitar
\renewcommand\numfrets{5}         % length (no of frets) of a diagram

% defines the chord above the lyrics
\renewcommand{\upchord}[1]{%
  \settowidth{\cwidth}{#1}%
  \raisebox{15pt}{\color{Gray}#1}\hspace{-\cwidth}%
}
\linespread{1.9}

%---

% % defines the chord within the lyrics
% \renewcommand{\upchord}[1]{[%
%   \settowidth{\cwidth}{#1}%
%   \raisebox{0pt}{\color{Gray}#1}%
% ]}
% \linespread{1}

%---

\newcommand\mychords{
  \def\chordsize{2.5mm}   % distance between two frets (and two strings)
  \font\fingerfont=cmr5  % font used for numbering fingers
  \font\fretposfont=cmr7  % font used for the fret position marker
  \def\dampsymbol{{\tiny$\scriptstyle\times$}} %  `damp this string' marker
}
\mychords

\renewcommand\yoff{3}
\renewcommand\fingsiz{1.6}

% Begin Chord definitions

% Major chords
\newcommand{\AflatMajor}{\marginpar{\chord{t}{5,3,4,3}{A$\flat$}}}
\newcommand{\Amajor}{\marginpar{\chord{t}{2,1,o,o}{A}}}
\newcommand{\AsharpMajor}{\marginpar{\chord{t}{3,2,1,1}{A$\sharp$}}}
\newcommand{\BflatMajor}{\marginpar{\chord{t}{3,2,1,1}{Bb}}}
\newcommand{\Bmajor}{\marginpar{\chord{t}{4,3,2,2}{B}}}
\newcommand{\Cmajor}{\marginpar{\chord{t}{o,o,o,3}{C}}}
\newcommand{\CmajorFirstInv}{\marginpar{\chord{t}{5,4,3,3}{C(1st)}}}
\newcommand{\CsharpMajor}{\marginpar{\chord{t}{1,1,1,4}{C$\sharp$}}}
\newcommand{\DflatMajor}{\marginpar{\chord{t}{1,1,1,4}{D$\flat$}}}
\newcommand{\Dmajor}{\marginpar{\chord{t}{2,2,2,o}{D}}}
\newcommand{\DmajorEasy}{\marginpar{\chord{t}{2,2,2,5}{D}}}
\newcommand{\DsharpMajor}{\marginpar{\chord{t}{3,3,3,6}{D$\sharp$}}}
\newcommand{\EflatMajor}{\marginpar{\chord{t}{3,3,3,6}{E$\flat$}}}
\newcommand{\Emajor}{\marginpar{\chord{t}{4,4,4,2}{E}}}
\newcommand{\EmajorEasy}{\marginpar{\chord{t}{4,4,4,7}{E}}}
\newcommand{\Fmajor}{\marginpar{\chord{t}{2,o,1,o}{F}}}
\newcommand{\FsharpMajor}{\marginpar{\chord{t}{3,1,2,1}{F$\sharp$}}}
\newcommand{\Gmajor}{\marginpar{\chord{t}{o,2,3,2}{G}}}
\newcommand{\GsharpMajor}{\marginpar{\chord{t}{5,3,4,3}{G$\sharp$}}}

% Minor chords
\newcommand{\Aminor}{\marginpar{\chord{t}{2,o,o,o}{A\large{m}}}}
\newcommand{\Bminor}{\marginpar{\chord{t}{4,2,2,2}{B\large{m}}}}
\newcommand{\Cminor}{\marginpar{\chord{t}{0,3,3,3}{C\large{m}}}}
\newcommand{\Dminor}{\marginpar{\chord{t}{2,2,1,o}{D\large{m}}}}
\newcommand{\Eminor}{\marginpar{\chord{t}{4,4,3,2}{E\large{m}}}}
\newcommand{\EminorEasy}{\marginpar{\chord{t}{o,4,3,2}{E\large{m}}}}
\newcommand{\Fminor}{\marginpar{\chord{t}{3,5,2,1}{F\large{m}}}}
\newcommand{\FsharpMinor}{\marginpar{\chord{t}{2,1,2,o}{F$\sharp$\large{m}}}}
\newcommand{\Gminor}{\marginpar{\chord{t}{o,2,3,1}{G\large{m}}}}


% Seventh chords
\newcommand{\Aseven}{\marginpar{\chord{t}{o,1,o,o}{A\large{7}}}}
\newcommand{\Cseven}{\marginpar{\chord{t}{o,o,o,1}{C\large{7}}}}
\newcommand{\CmajorSeven}{\marginpar{\chord{t}{o,o,o,2}{Cmaj\large{7}}}}
\newcommand{\Dseven}{\marginpar{\chord{t}{2,2,2,3}{D\large{7}}}}
\newcommand{\DminorSeven}{\marginpar{\chord{t}{2,2,1,3}{D\large{m}7}}}
\newcommand{\Eseven}{\marginpar{\chord{t}{1,2,o,2}{E\large{7}}}}
\newcommand{\EsevenFirstInv}{\marginpar{\chord{t}{4,4,4,5}{E\large{7}(1st)}}}
\newcommand{\EminorSeven}{\marginpar{\chord{t}{o,2,o,2}{E\large{m}7}}}
\newcommand{\FmajorSeven}{\marginpar{\chord{t}{2,4,1,o}{E\large{m}7}}}
\newcommand{\Gseven}{\marginpar{\chord{t}{o,2,1,2}{G\large{7}}}}

% Other chords
\newcommand{\Bflat}{\marginpar{\chord{t}{3,2,1,1}{B$\flat$}}}
\newcommand{\BflatDiminishedSeven}{\marginpar{\chord{t}{3,2,1,1}{B$\flat$dim\large{7}}}}
\newcommand{\Caugmented}{\marginpar{\chord{t}{1,o,o,3}{C+}}}
\newcommand{\EsevenSusFour}{\marginpar{\chord{t}{2,2,o,2}{C+}}}

\newcommand{\Gaugmented}{\marginpar{\chord{t}{4,3,3,2}{G+}}}

% End Chord definitions



% ------------------- Title and Author -----------------------------
\title{Balham Ukulele Society Songbook}
\author{Maintainer: Vish Vishvanath}
\begin{document}

\pagenumbering{roman}
\pagestyle{plain}
\newgeometry{top=2cm, bottom=2cm, left=2in, right=2in, textwidth=17cm, 
       textheight=26cm, marginparwidth=0in, nofoot, centering}
\maketitle
\restoregeometry

\dominitoc
\dominilof
\dominilot
\tableofcontents

\pagenumbering{arabic}
\setcounter{page}{1}

% \LARGE

\chapter{Introduction}\label{ch:introduction}
\section{Balham Ukulele Society} % (fold)
\label{sec:balham_ukulele_society}

\paragraph{Welcome} % (fold)
\label{par:welcome}
Balham Ukulele Society are nominally based at the Balham Bowls Club, one of the most wonderful pubs around, quirky, capacious, tolerant and friendly, with decent beer. This is where we meet for a jam session every other Sunday at 7pm. This is part of the wider Songbook project.
% paragraph welcome (end)

\paragraph{Contact Us} % (fold)
\label{par:contact_us}
The organizers are available in the pub, or at \href{mailto:balhamukulelesociety@gmail.com}{balhamukulelesociety@gmail.com}
% paragraph contact_us (end)

% section balham_ukulele_society (end)


\chapter{Songs / A -- M}
\label{ch:songs_a_m}
\minitoc
\Large
\linenumbers
\modulolinenumbers[2]

%% START SONGS AM
\section{Ace Of Spades / Motorhead}\label{sec:ace_of_spades}
{\small (Really helps to play this with barre chords but your fingers will still get tired.)}
\Bmajor
\Cmajor
\Dmajor
\Emajor
\Gmajor

\upchord{E}Intro\\

\upchord{G}If you like to gamble, I tell you I'm your man

\upchord{G}You win some, lose some, it's all the same to me\upchord{E}

\upchord{D}The pleasure is to \upchord{C}play, it makes no difference what you say\upchord{E}

\upchord{D}I don't share your \upchord{C}greed, the only card I need is\\

(x2)\upchord{E}The Ace Of Spades

Alright\\

\upchord{G}Playing for the high one, dancing with the devil,

\upchord{G}Going with the flow, it's all a game to me\upchord{E}

Sev\upchord{D}en or \upchord{C}Eleven, snake eyes watching you\upchord{E}

\upchord{D}Double up or \upchord{C}quit, double stakes or splits\\

(x2)\upchord{E}The Ace Of Spades\\

\upchord{E}You know I'm born to lose, and \upchord{D}gamb\upchord{E}ling's for fools,

\upchord{E}But that's the way I like it babe

\upchord{E}I don't wanna live forever \upchord{D}\hrulefill\upchord{E}\hrulefill\upchord{D}\hrulefill\upchord{C}\hrulefill\upchord{B}\hrulefill

\upchord{B}And Don't Forget The Joker\upchord{E}\\

\upchord{G}Pushing up the ante, I know you've got to see me

\upchord{G}Read 'em and weep, the dead man's hand again\upchord{E}

\upchord{D}I see it in your \upchord{C}eyes, take one look and die\upchord{E}

\upchord{D}The only thing you \upchord{C}see, you know it's gonna be\\

(x2)\upchord{E}The Ace Of Spades



\section{All My Loving / The Beatles}\label{sec:all_my_loving}
{\small (Play Am/C by fingering both chords simultaneously. 3rd fret 1st string + 2nd fret 4th string)}

\Cmajor
\Gmajor
\Dminor
\Gseven
\Aminor
\Fmajor
\Bflat
\Caugmented

\upchord{C}Intro:\upchord{G}\hrulefill\upchord{C}\hrulefill

Close your \upchord{Dm}eyes and I'll \upchord{G7}kiss you

To\upchord{C}morrow I'll \upchord{Am}miss you

Re\upchord{F}member I'll \upchord{Dm}always be \upchord{B$\flat$}true 

\upchord{G7}And then \upchord{Dm}while I'm a\upchord{G7}way

I'll write \upchord{C}home every \upchord{Am}day

And I'll \upchord{F}send all my \upchord{G7}loving to \upchord{C}you\\

I'll pre\upchord{Dm}tend that I'm \upchord{G7}kissing

The \upchord{C}lips I am \upchord{Am}missing

And \upchord{F}hope that my \upchord{Dm}dreams will come \upchord{B$\flat$}true

\upchord{G7}And then \upchord{Dm}while I'm \upchord{G7}away

I'll write \upchord{C}home ev'ry \upchord{Am}day

And I'll \upchord{F}send all my \upchord{G7}loving to \upchord{C}you\\

\upchord{C}All my \upchord{Am/C}loving | \upchord{C+}I will send to \upchord{C}you 

\upchord{C}All my \upchord{Am/C}loving | \upchord{C+}darling I'll be \upchord{C}true\\

Instrumental: \upchord{F}\hrulefill\upchord{C}\hrulefill\upchord{Dm}\hrulefill\upchord{G7}\hrulefill\upchord{C}\hrulefill\\

Close your \upchord{Dm}eyes and I'll \upchord{G7}kiss you

To\upchord{C}morrow I'll \upchord{Am}miss you

Re\upchord{F}member I'll \upchord{Dm}always be \upchord{B$\flat$}true 

\upchord{G7}And then \upchord{Dm}while I'm a\upchord{G7}way

I'll write \upchord{C}home ev'ry \upchord{Am}day

And I'll \upchord{F}send all my \upchord{G7}loving to \upchord{C}you\\


All my \upchord{Am/C}loving | \upchord{C+}I will send to \upchord{C}you 

All my \upchord{Am/C}loving | \upchord{C+}darling I'll be \upchord{C}true 

All my \upchord{Am} loving all my \upchord{C}loving ooh

All my \upchord{Am}loving | I will send to \upchord{C}you

\section{Alright / Supergrass}\label{sec:alright}
  {\small chords}
  
  \Eminor
  \Fsharpminor
  \Fmajor
  \Amajor
  \Gmajor
  
  \upchord{D}We are young, we run green, keep our teeth, nice and clean,\\
  See our \upchord{Em}friends, see the sights, feel \upchord{D}alright,\\
  We wake up, we go out, smoke a fag, put it out,\\
  See our \upchord{Em}friends, see the sights, feel \upchord{D}alright,\\
  \upchord{F\#m}Are we like you? \upchord{F}I can't be sure,\\
  Of the \upchord{Em}scene, as she turns, we are \upchord{A}strange in our worlds\\
  But we are \upchord{D}young, we get by, can't go mad, ain't got time,\\
  Sleep \upchord{Em}around, if we like, but we're \upchord{D}alright.\\
  Got some cash, bought some wheels, took it out, 'cross the fields,\\
  Lost \upchord{Em}control, hit a wall, But we're \upchord{D}alright.\\
  \upchord{F\#m}Are we like you, I \upchord{F}can't be sure,\\
  Of the \upchord{Em}scene, as she turns, we are \upchord{A}strange in our worlds\\
  But we are \upchord{D}young, we run green, keep our teeth, nice and clean,\\
  See our \upchord{Em}friends, see the sights, feel \upchord{D}alright,\\
  Instrumental:\\
  \upchord{G} / \upchord{F} / \upchord{G} / \upchord{F} / \upchord{G} / \upchord{F} / \upchord{Em} / \upchord{A}\\
  \upchord{D} / \upchord{D} / \upchord{D} / \upchord{D} / \upchord{Em} / \upchord{Em} / \upchord{D} / \upchord{D}\\
  \upchord{F\#m}Are we like you, I \upchord{F}can't be sure,\\
  Of the \upchord{Em}scene, as she turns, we are \upchord{A}strange in our worlds\\
  But we are \upchord{D}young, we run green, keep our teeth, nice and clean,\\
  See our \upchord{Em}friends, see the sights, feel \upchord{D}alright. CHA CHA CHA.\\

\section{Dirty Old Town / Ewan MacColl}\label{sec:dirty_old_town}
Info\footnote{You shouldn’t really need telling how to play this one. Note the unusual key differences: D, G and C). (For anyone who didn’t know Ewan MacColl was Kirsty MacColl's dad--Ewan MacColl was Kirsty MacColl's dad)}
\Cmajor
\Dmajor
\Fmajor
\Gmajor
\Aminor
\Bminor
\EminorSeven

\upchord{D}Intro:\upchord{G}\hrulefill\upchord{D}\hrulefill\upchord{Em7}\hrulefill\upchord{Bm}\hrulefill

I met my \upchord{G}love by the gas works wall 

Dreamed a \upchord{C}dream by the old ca\upchord{G}nal

I kissed my girl by the factory wall

Dirty old \upchord{D}town

Dirty old \upchord{Em7}town\\


Clouds are \upchord{G}drifting across the moon 

Cats are \upchord{C}prowling on their \upchord{G}beat 

Spring's a girl from the streets at night 

Dirty old \upchord{D}town | Dirty old \upchord{Em7}town\\


Instrumental: \upchord{C}\hrulefill\upchord{F}\hrulefill\upchord{C}\hrulefill\upchord{G}\hrulefill\upchord{Am}\hrulefill\\


I heard a \upchord{G}siren from the docks

Saw a \upchord{C}train set the night on \upchord{G}fire 

I smelled the spring on the smoky wind 

Dirty old \upchord{D}town

Dirty old \upchord{Em7}town\\


I'm gonna \upchord{G}make me a big sharp axe 

Shining \upchord{C}steel tempered in the \upchord{G}fire 

I'll chop you down like an old dead tree 

Dirty old \upchord{D}town

Dirty old \upchord{Em7}town\\


I met my \upchord{G}love by the gas works wall 

Dreamed a \upchord{C}dream by the old ca\upchord{G}nal

I kissed my girl by the factory wall

Dirty old \upchord{Am}town

Dirty old \upchord{Em7}town

Dirty old \upchord{D}town

Dirty old \upchord{Em7}town
\section{Downtown / Tony Hatch}\label{sec:downtown}
  {\small chords}
  
  \Aminor
  \BflatMajor
  \Cseven
  \Dminor
  \Gseven
  
  Verse
  \upchord{F}When you're a\upchord{Am}lone, And life is \upchord{Bb}making you \upchord{C7}lonely,
  You can \upchord{F}always \upchord{Bb}go \upchord{C7}downtown
  \upchord{F}When you've got \upchord{Am}worries, All the \upchord{Bb}noise and the \upchord{C7}hurry
  Seems to \upchord{F}help, I \upchord{Am}know,\upchord{Bb} down\upchord{C7}town
  Bridge
  Just \upchord{F}listen to the music of the \upchord{Dm}traffic in the city
  \upchord{F}Linger on the sidewalk where the \upchord{Dm}neon signs are pretty, \upchord{C}How can you lose?
  \upchord{Bb}The lights are much brighter there
  You can for-\upchord{G7}get all your troubles, forget all your cares and go
  Chorus
  \upchord{F}Down\upchord{Am}town, \upchord{Bb}things'll be \upchord{C7}great when you're
  \upchord{F}Down\upchord{Am}town, \upchord{Bb}no finer \upchord{C7}place for sure
  \upchord{F}Down\upchord{Am}town, \upchord{Bb}everything's \upchord{C7}waiting for you
  (Downtown)
  \upchord{F}Don't hang a\upchord{Am}round And let your \upchord{Bb}problems sur\upchord{C7}round you
  There are \upchord{F}movie \upchord{Am}shows
  \upchord{Bb}down\upchord{C7}town
  \upchord{F}Maybe you \upchord{Am}know Some little \upchord{Bb}places to \upchord{C7}go to
  Where they \upchord{F}never \upchord{Am}close \upchord{Bb}down\upchord{C7}town
  Just \upchord{F}listen to the rhythm of a \upchord{Dm}gentle bossanova
  \upchord{F}You'll be dancing with 'em too \upchord{Dm}before the night is over, \upchord{C}Happy again
  \upchord{Bb}The lights are much brighter there
  You can for\upchord{G7}get all your troubles, forget all your cares and go
  \upchord{F}Down\upchord{Am}town, \upchord{Bb}where all the \upchord{C7}lights are bright,
  

\section{Echo Beach / Martha and the Muffins}\label{sec:echo_beach}
\Cmajor
\Dmajor
\Fmajor
\Gmajor
\Aminor
\Eminor
\Bflat

\upchord{Am}Intro:\hrulefill\upchord{G}\hrulefill\upchord{Em}\hrulefill\upchord{F}\hrulefill\upchord{G}\hrulefill x 4

I \upchord{Am}know it's out of fashion, \upchord{D}and a \upchord{C}trifle \upchord{Am}uncool\upchord{D}\hrulefill\upchord{Em}\hrulefill

But \upchord{Am}I can't help it, \upchord{D}I'm a \upchord{C}romantic \upchord{Am}fool\upchord{D}\hrulefill\upchord{Em}\hrulefill

It's a \upchord{Am}habit of mine \upchord{D}to watch the \upchord{C}sun go \upchord{Am}down\upchord{D}\hrulefill

\upchord{Em}On \upchord{Am}Echo Beach, \upchord{D}I watch the \upchord{C}sun go \upchord{Am}down\upchord{D}\hrulefill\upchord{Em}\hrulefill\\

From \upchord{G}9 to 5 I have to spend my \upchord{D}time at work.

My \upchord{G}job is very boring, I'm an \upchord{D}office clerk.

The \upchord{Am}only thing that helps me pass the \upchord{Em}time away

Is \upchord{Am}knowing I'll be back in Echo \upchord{Em}Beach some day.\\

Instrumental:\upchord{Am}\hrulefill\upchord{G}\hrulefill\upchord{Em}\hrulefill\upchord{F}\hrulefill\upchord{G}\hrulefill x 4\\

On \upchord{Am}silent summer evenings \upchord{D}the sky's \upchord{C}alive with \upchord{Am}light\upchord{D}\hrulefill\upchord{Em}\hrulefill

A \upchord{Am}building in the distance - \upchord{D}surreal\upchord{C}istic \upchord{Am}sight\upchord{D}\hrulefill

On \upchord{Am}Echo Beach, \upchord{D}waves make the \upchord{C}only \upchord{Am}sound\upchord{D}\hrulefill

On \upchord{Am}Echo Beach, \upchord{D}there's not a \upchord{C}soul a\upchord{Am}round\upchord{D}\hrulefill\upchord{Em}\hrulefill\\

From \upchord{G}9 to 5 I have to spend my \upchord{D}time at work.

My \upchord{G}job is very boring, I'm an \upchord{D}office clerk.

The \upchord{Am}only thing that helps me pass the \upchord{Em}time away

Is \upchord{Am}knowing I'll be back in Echo \upchord{Em}Beach some day.\\

\upchord{F}\hrulefill\upchord{G}\hrulefill\upchord{B$\flat$}\hrulefill\upchord{C}\hrulefill x 2\\

\upchord{Am}Echo Beach \upchord{G}far away in time, \upchord{Em}Echo Beach \upchord{F}far away \upchord{G}in time...
(repeat to end)
\section{Five Foot Two / Ray Henderson}\label{ch:five_foot_two}
\Cmajor
\Aseven
\Dseven
\Eseven
\Gseven
\Gaugmented

\upchord{C}Five foot two, \upchord{E7}eyes of blue

\upchord{A7}But oh what those five foot could do

Has \upchord{D7}anybody \upchord{G7}seen my \upchord{C}girl?\\


\upchord{C}Turned up nose, \upchord{E7}turned down hose

\upchord{A7}Never had no other beaus

Has \upchord{D7}anybody \upchord{G7}seen my \upchord{C}girl?\\


\upchord{E7}Now if you run into a five foot two

\upchord{A7}All covered in fur

\upchord{D7}Diamond rings and all those things

\upchord{G7}Betcha life \upchord{D7}it isn't \upchord{G7}her\\


\upchord{G+}But

\upchord{C}Could she love, \upchord{E7}could she woo?

\upchord{A7}Could she, could she, could she coo?

Has \upchord{D7}anybody \upchord{G7}seen my \upchord{C}girl?

\section{Hallelujah / Leonard Cohen}\label{sec:hallelujah}
\Cmajor
\Fmajor
\Gmajor
\Aminor
\Eseven

\upchord{C}Intro:\upchord{Am}\hrulefill\upchord{C}\hrulefill\upchord{Am}

I \upchord{C}heard there was a \upchord{Am}secret chord

that \upchord{C}David played and it \upchord{Am}pleased the lord

but \upchord{F}you don't really \upchord{G}care for music \upchord{C}do you \upchord{G}

Well it \upchord{C}goes like this the \upchord{F}fourth, the \upchord{G}fifth a the \upchord{Am}minor fall and the \upchord{F}major lift | the \upchord{G}baffled king \upchord{E7}composing \upchord{Am}hallelujah

Halle-\upchord{F}lujah, Halle-\upchord{Am}lujah, Halle-\upchord{F}lujah, Hallelu-\upchord{C}\upchord{G}jah \upchord{C}\hrulefill\upchord{Am}\hrulefill\upchord{C}\hrulefill\upchord{Am}\hrulefill

Well your \upchord{C}faith was strong but you \upchord{Am}needed proof

You \upchord{C}saw her bathing \upchord{Am}on the roof

Her \upchord{F}beauty and the \upchord{G}moonlight overthrew \upchord{C}you \upchord{G}

She \upchord{C}tied you to her \upchord{F}kitchen \upchord{G}chair, she \upchord{Am}broke your throne and she \upchord{F}cut your hair | And \upchord{G}from your lips she \upchord{E7}drew the \upchord{Am}hallelujah

\upchord{C}Baby i've been \upchord{Am}here before I've \upchord{C}seen this room and I've \upchord{Am}walked this floor

I \upchord{F}used to live \upchord{G}alone before I \upchord{C}knew you \upchord{G}

I've \upchord{C}seen your flag on the \upchord{F}marble \upchord{G}arch

But \upchord{Am}love is not a \upchord{F}victory march

It's a \upchord{G}cold and it's a \upchord{E7}broken \upchord{Am}hallelujah

Well \upchord{C}there was a time when you \upchord{Am}let me know 

What's \upchord{C}really going \upchord{Am}on below

But \upchord{F}now you never \upchord{G}show that to me \upchord{C}do you \upchord{G}\hrulefill

But \upchord{C}remember when I \upchord{F}moved in \upchord{G}you | And the \upchord{Am}holy dove was \upchord{F}moving too

And \upchord{G}every breath we \upchord{E7}drew was \upchord{Am}hallelujah

Well \upchord{C}maybe there's a \upchord{Am}god above | But \upchord{C}all I've ever \upchord{Am}learned from love

Was \upchord{F}how to shoot \upchord{G}somebody who \upchord{C}outdrew you | \upchord{G}It's \upchord{C}not a cry that you \upchord{F}hear at \upchord{G}night

It's \upchord{Am}not somebody who's \upchord{F}seen the light | It's a \upchord{G}cold and it's a \upchord{E7}broken \upchord{Am}hallelujah

\upchord{F}Hallelujah, \upchord{Am}Hallelujah, \upchord{F}Hallelujah, \upchord{C}Hallel\upchord{G}ujah \upchord{C}\hrulefill
\section{Hounds of Love / Kate Bush}\label{sec:hounds_of_love}
  {\small We need different people doing the doos, ows, and in-the-trees from those singing the bridge parts. The rhythm in this is a bit of a fucker. We'll probably need someone banging a drum.}
  
  \Fmajor
  \Cmajor
  \BflatMajor
  \Aminor
  \DminorSeven
  
  It's in the trees!
  It's coming!\\
  Chorus:\\
  \upchord{F}When I was a \upchord{C}child:
  running in the \upchord{Bb}night,\\
  Af\upchord{Bb}raid of what might \upchord{F}be -
  hiding in the \upchord{C}dark,\\
  Hiding in the \upchord{Bb}street,
  And of \upchord{Bb}what was following \upchord{Dm}me...\\
  Verse:\\
  doo doo \upchord{C}doo doo doo\upchord{Bb}
  now hounds of \upchord{Am}love are haunting \upchord{Dm}me. \\
  oo oo\upchord{Bb} oo oo oo \upchord{Am}
  I've always \upchord{C}been a coward\upchord{Dm},\\
  oo oo\upchord{Bb} oo oo oo \upchord{Am}
  and I don't \upchord{C}know what's good for me.\upchord{Dm}\\
  Bridge:\\
  \upchord{Dm}
  Here I \upchord{F}go!
  \upchord{Dm7}It's coming \upchord{Bb}for me through the \upchord{Bb}trees.\upchord{F}\\
  Help me, \upchord{Dm7}someone!
  help me, \upchord{Bb}please!\upchord{Bb}\\
  \upchord{F}Take my \upchord{Dm7}shoes off,
  and \upchord{Bb}throw them in the lake\upchord{Bb},\\
  And I'll \upchord{F}be
  \upchord{Dm7}two steps on the \upchord{Bb}water.\upchord{Bb}\\
  \upchord{F}I found a \upchord{C}fox
  caught by \upchord{Bb}dogs.\\
  He let me \upchord{Bb}take him in my \upchord{F}hands.
  his little \upchord{C}heart,\\
  It beat so \upchord{Bb}fast,
  and I'm \upchord{Bb}ashamed of running \upchord{F}away\\
  From nothing \upchord{C}real--
  I just can't \upchord{Bb}deal with this,\\
  But \upchord{Bb}I'm still afraid to \upchord{Dm}be there,
  doo doo\upchord{C} doo doo doo doo\upchord{Bb}\\
  Among your \upchord{Am}hounds of love,\upchord{Dm}
  ow ow\upchord{Bb} ow ow ow ow\upchord{Am}\\
  And feel your \upchord{C}arms surrounding \upchord{Dm}me.
  ow ow \upchord{Bb}ow ow ow ow\upchord{Am}\\
  I've always \upchord{C}been a coward\upchord{Dm},\\
  \upchord{Bb}And never \upchord{Am}know what's good for \upchord{C}me.\upchord{Dm}\\
  \upchord{Dm}
  Oh, here I \upchord{F}go!
  \upchord{Dm7}Don't let me \upchord{Bb}go!\\
  \upchord{Bb}Hold me \upchord{F}down!
  It's \upchord{Dm7}coming for me \upchord{Bb}through the trees.\upchord{Bb}\\
  \upchord{F}Help me, \upchord{Dm7}darling,
  Help me, \upchord{Bb}please!\upchord{Bb}\\
  \upchord{F}Take my \upchord{Dm7}shoes off
  and \upchord{Bb}throw them in the \upchord{Bb}lake,\\
  And I'll be\upchord{F}\\
  \upchord{Dm7}Two steps on the \upchord{Bb}water. \upchord{Bb}\\
  \upchord{F}I don't know what's \upchord{Dm7}good for me.\\
  I don't know what's \upchord{Bb}good for me.\\
  I need your \upchord{Bb}la la la la la la, \upchord{F}yeahie yo \upchord{Dm7}yeahie yo !
  \upchord{Bb}Your love!\\
  do do do\upchord{Bb}\\
  \upchord{F}Take your \upchord{Dm7}shoes off\\
  And \upchord{Bb}throw them in the \upchord{Bb}lake!
  \upchord{F}Do you know what I \upchord{Dm7}really need?\\
  Do you know what I \upchord{Bb}really need?\\
  I need \upchord{Bb}love love love love love, \upchord{F}yeah!\\

\section{I Will Survive / Gloria Gaynor}\label{sec:i_will_survive}
  {\small chords}
  
  \Dminor
  \Gseven
  \CmajorSeven
  \FmajorSeven
  \EsevenSusFour
  \Eseven
  
  \upchord{Am} At first I was afraid I was \upchord{Dm} petrified
  Kept thinkin' \upchord{G7} I could never live without you \upchord{Cmaj7} by my side
  But then I \upchord{Fmaj7} spent so many nights thinkin'
  \upchord{Dm} How you did me wrong
  I grew \upchord{E7sus4} strong I learned \upchord{E7} how to get along
  But now you're \upchord{Am} back
  from outer \upchord{Dm} space
  I just walked \upchord{G7} in to find you here
  With that sad \upchord{Cmaj7} look upon your face
  I should have \upchord{Fmaj7} changed that stupid lock
  I should have \upchord{Dm} made you leave your key
  If I'd've \upchord{E7sus4} known for just one second
  You'd be \upchord{E7} back to bother me
  Chorus:
  Go on now \upchord{Am} go
  walk out the \upchord{Dm} door
  Just turn a\upchord{G7}round now 'cause you're not \upchord{Cmaj7} welcome anymore
  \upchord{Fmaj7} Weren't you the one who tried to \upchord{Dm} hurt me with goodbye
  Did you think I’d \upchord{E7sus4} crumble did you think I'd \upchord{E7} lay down and die
  Oh no not \upchord{Am} I I will sur\upchord{Dm}vive
  Oh as \upchord{G7} long as I know how to love I \upchord{Cmaj7} know I'll stay alive
  I've got \upchord{Fmaj7} all my life to live and I've got \upchord{Dm} all my love to give
  And I'll sur\upchord{E7sus4}vive I will sur\upchord{E7}vive
  I will sur\upchord{Am}vive \upchord{Dm} \upchord{G7} \upchord{Cmaj7} \upchord{Fmaj7} \upchord{Dm} \upchord{E7sus4} \upchord{E7}
  It took \upchord{Am} all the strength I had not to \upchord{Dm} fall apart
  Though I tried \upchord{G7} hard to mend the pieces of my \upchord{Cmaj7} broken heart
  And I spent \upchord{Fmaj7} oh so many nights just feeling \upchord{Dm} sorry for myself
  I used to \upchord{E7sus4} cry but now I \upchord{E7} hold my head up high
  And you see \upchord{Am} me somebody \upchord{Dm} new
  I'm not that \upchord{G7} chained up little person still in \upchord{Cmaj7} love with you
  And so you \upchord{Fmaj7} felt like droppin' in and just ex\upchord{Dm}pect me to be free
  Now I'm \upchord{E7sus4} savin' all my lovin' for some\upchord{E7}one who's lovin' me
  Chorus
  Repeat Chorus

\section{Justified and Ancient / The KLF}\label{sec:justified_and_ancient}
  {\small chords}
  
  \Eminor
  \Dmajor
  \FsharpMinor
  \Gmajor
  \Amajor
  
  \upchord{Bm} All bound for Mu Mu Land, All bound for Mu Mu Land
  \upchord{Bm} \upchord{riff}
  \upchord{Em} All bound for Mu Mu Land
  \upchord{Bm} \upchord{riff}
  \upchord{Em} All bound for Mu Mu Land
  (Bring the beat back!)
  They're \upchord{D} Justified, and they’re \upchord{F#m} Ancient,
  And they \upchord{G} like to roam the \upchord{A} land.
  They're \upchord{D} Justified, and they're \upchord{F#m} Ancient,
  I \upchord{G} hope you under-\upchord{A}-stand.
  They \upchord{G} called me up in \upchord{D} Tennessee
  They said \upchord{G} "Tammy, stand by The \upchord{D} Jams"
  But \upchord{G} if you don't like what they're \upchord{D} going to do,
  You \upchord{A} better not stop them 'cause they're coming through
  \upchord{Bm} \upchord{riff} (Hey hey)
  \upchord{Em} All bound for Mu Mu Land (justified)
  \upchord{Bm} \upchord{riff} (Hey hey)
  \upchord{Em} All bound for Mu Mu Land (justified)
  (Ancients of Mu Mu) (?)
  They're \upchord{D} Justified, and they're \upchord{F#m} Ancient,
  And they \upchord{G} drive an ice cream \upchord{A} van.
  (just roll it from the top)
  They're \upchord{D} Justified and they're \upchord{F#m} Ancient,
  With \upchord{G} still no master \upchord{A} plan.
  The \upchord{G} last train left an \upchord{D} hour ago,
  They were \upchord{G} singing "All a-\upchord{D}-board"
  \upchord{G} All bound for \upchord{D} Mu Mu Land,
  Then \upchord{A} someone starting screaming "Turn up the Strobe"
  (bring the beat back)
  \upchord{Bm} \upchord{riff} (Hey hey)
  \upchord{Em} All bound for Mu Mu Land (justified)
  \upchord{Bm} \upchord{riff} (Hey hey)
  \upchord{Em} All bound for Mu Mu Land
  (Ancients of Mu Mu) (?)
  \upchord{Bm} Justified and Ancient, Ancient and a-justified,
  Rocking to the rhythm in their ice cream van
  with the plan and the key to
  enter into Mu Mu
  Vibes from the tribes of the Jams
  I know where the beat is at,
  'cos I know what time it is
  Bring home a dime,
  Make mine a "99"
  New style, meanwhile, always on a mission while
  Fishing in the rivers of life
  Fishing in the rivers of life (HOI)
  Fishing in the rivers of life (HOI)
  Fishing in the rivers
  Fishing in the rivers
  Fishing in the rivers of life (HOI)
  Voo-va-voolie
  Za-shi-va-zom
  \upchord{Em} Voo-va-voolie (pause)
  (BRING THE BEAT BACK)
  \upchord{Bm} \upchord{riff} (Hey hey)
  \upchord{Em} All bound for Mu Mu Land
  \upchord{Bm} \upchord{riff} (Hey hey)
  \upchord{Em} All bound for Mu Mu Land
  \upchord{single strums - Bm} Mu Mu Land, Mu Mu Land (ANCIENTS OF MU MU, ANCIENTS OF MU MU)
  \upchord{Em} All bound for Mu Mu Land
  \upchord{Bm} Mu Mu Land, Mu Mu Land (ANCIENTS OF MU MU, ANCIENTS OF MU MU)
  \upchord{Em} All bound for Mu Mu Land
  \upchord{a cappella} Mu Mu Land, Mu Mu Land
  All bound for Mu Mu Land


%% END SONGS AM

\nolinenumbers
% chapter songs_a_m (end)

\chapter{Songs / N -- Z} % (fold)
\label{cha:songs_n_z}
\minitoc
\Large
\linenumbers
\modulolinenumbers[2]

%% START SONGS NZ
\section{Radio Ga Ga / Queen}\label{sec:radio_ga_ga}
  {\small / symbols in chorus denote knocking uke. Loads of percussive opportunities here. Probably best during intro and choruses to have three parts: 1 purely percussive and two different strumming. We’ll sort this out at the meeting.}
  
  \Fmajor
  \BflatMajor
  \Gminor
  \Cmajor
  
  \upchord{F} \upchord{Gm} \upchord{Bb} \upchord{Gm} \upchord{Bb} \upchord{F}
  \upchord{F} \upchord{Gm} \upchord{Bb} \upchord{Gm} \upchord{Bb} \upchord{F} \upchord{Bb} \upchord{F} \upchord{Bb}
  \upchord{F}I'd sit alone and watch your light\\
  My \upchord{Gm}only friend through teenage nights\\
  And \upchord{Bb}everything I had to know\\
  I \upchord{Gm}heard it on my \upchord{Bb}ra-\upchord{F}dio\\
  You \upchord{F}gave them all those old time stars\\
  Through \upchord{Gm}wars of worlds - invaded by Mars\\
  You \upchord{Bb}made 'em laugh - you made 'em cry\\
  You \upchord{Gm}made us feel like we c\upchord{Bb}ould \upchord{F}fly (\upchord{Bb}ra - \upchord{F}dio)\\
  So \upchord{F}don't become some background noise\\
  A \upchord{Abdim}backdrop for the girls and boys\\
  Who \upchord{Bb}just don't know or just don't care\\
  And \upchord{G}just complain when you're not there\\
  You \upchord{F}had your time, you had the power\\
  You've \upchord{C}yet to have your finest hour\\
  \upchord{Bb}Ra – \upchord{F}dio, (\upchord{Bb}ra – \upchord{F}dio)\\
  CHORUS:\\
  \upchord{F7}All we hear is \upchord{Bb}radio \upchord{F}ga ga / /\\
  \upchord{Bb}radio \upchord{F}goo goo / / \upchord{Bb}radio \upchord{F}ga ga / /\\
  \upchord{F7}All we hear is \upchord{Bb}radio \upchord{F}ga ga / /\\
  \upchord{Bb}radio \upchord{F}blah blah / /\\
  \upchord{Eb}Radio what's \upchord{Bb}new\upchord{C}?\\
  \upchord{Dm}Radio, \upchord{C}someone still loves\upchord{F} you\\
  We \upchord{F}watch the shows - we watch the stars\\
  On \upchord{Gm}videos for hours and hours\\
  We \upchord{Bb}hardly need to use our ears\\
  How \upchord{Gm}music changes thro\upchord{Bb}ugh the \upchord{F}years\\
  \upchord{F}Let's hope you never leave old friend\\
  Like \upchord{Abdim}all good things on you we depend\\
  So \upchord{Bb}stick around 'cos we might miss you\\
  When \upchord{G7}we grow tired of all this visual\\
  You \upchord{F}had your time - you had the power\\
  You've \upchord{C}yet to have your finest hour\\
  \upchord{Bb}Ra – \upchord{F}dio, (\upchord{Bb}ra – \upchord{F}dio)\\
  (REPEAT CHORUS)\\

\section{The Shoop Shoop Song / Rudy Clark}\label{sec:shoop_shoop_song}
% {\small(Girls: don't sing the parts in italics -- Guys: italics are for you)}
\Cmajor
\Fmajor
\Gmajor
\Aminor
\Dminor
\Dseven
\Eseven


\upchord{G}Does he love me \upchord{F}I want to know\\
\upchord{G}How can I tell if he loves me so\\
\emph{(Is it \upchord{Dm}in his \upchord{G}eyes?)} Oh \upchord{Dm}no, you'll be de\upchord{G}ceived\\
\emph{(Is it \upchord{Dm}in his \upchord{G}eyes?)} Oh \upchord{Dm}no he'll make be\upchord{G}lieve\\
If you \upchord{C}wanna \upchord{Am}know if \upchord{F}he loves you \upchord{G}so, it's in his \upchord{C}kiss\\
\emph{\upchord{F}(That's where it \upchord{G}is)} ...Oh yeah\\
\emph{(Is it \upchord{Dm}in his \upchord{G}face?)} Oh \upchord{Dm}no, that's just his \upchord{G}charm\\
\emph{(In his \upchord{Dm}warm em\upchord{G}brace?)} Oh \upchord{Dm}no, that's just his \upchord{G}arms\\
If you \upchord{C}wanna \upchord{Am}know if \upchord{F}he loves you \upchord{G}so, it's in his \upchord{C}kiss \\
\upchord{F}(That's where it \upchord{G}is) ...Oh yeah it's in his \upchord{C}kiss\\
\emph{\upchord{F}(That's where it \upchord{C}is)} Oh, Oh, Oh,\\
\upchord{E7}hug him, squeeze him tight\\
To \upchord{Am}find out what you want to know \upchord{D7}If it's love, if it really is\\
\upchord{G}It's there in his kiss\\
\emph{(How 'bout the \upchord{Dm}way he \upchord{G}acts)}\\
Oh \upchord{Dm}no, that's not the \upchord{G}way\\
\upchord{Dm}You're not \upchord{G}listening to \upchord{Dm}all I \upchord{G}say\\
If you \upchord{C}wanna \upchord{Am}know if \upchord{F}he loves you \upchord{G}so, it's in his \upchord{C}kiss\\
\emph{\upchord{F}(That's where it \upchord{G}is)} oh, oh, it's in his \upchord{C}kiss\\
\emph{\upchord{F}(That's where it \upchord{G}is)}\\
Instrumental:\upchord{Dm}\hrulefill\upchord{G}\hrulefill\upchord{Dm}\hrulefill\upchord{G}\hrulefill\upchord{C}\hrulefill x2\\
\upchord{E7}hug him, squeeze him tight\\
To \upchord{Am}find out what you want to know \upchord{D7}If it's love, if it really is\\
\upchord{G}It's there in his kiss\\
\emph{(How 'bout the \upchord{Dm}way he \upchord{G}acts)} oh \upchord{Dm}no, that's not the \upchord{G}way\\
\upchord{Dm}You're not \upchord{G}listening to \upchord{Dm}all I \upchord{G}say\\
If you \upchord{C}wanna \upchord{Am}know if \upchord{F}he loves you \upchord{G}so, it's in his \upchord{C}kiss\\
\emph{\upchord{F}(That's where it \upchord{G}is)} oh, oh, it's in his \upchord{C}kiss\\
\emph{\upchord{F}(That's where it \upchord{G}is)} oh, oh, it's in his \upchord{C}kiss\\
\emph{\upchord{F}(That's where it \upchord{G}is)} oh yeah, it's in his kiss\\
\upchord{C}\hrulefill\upchord{F}\hrulefill\upchord{C}

\section{Sloop John B / Beach Boys}\label{sec:sloop_john_b}
  {\small chords}
  
  \Cmajor
  \Gseven
  \Dseven
  \Aminor
  
  \upchord{G} We come on the sloop John B\\
  My grandfather and me\\
  Around Nassau town we did \upchord{D7} roam\\
  Drinking all \upchord{G} night \upchord{G7} got into a \upchord{C} fight \upchord{Am}\\
  Well I \upchord{G} feel so broke up \upchord{D7} I want to go \upchord{G} home\\
  Chorus:\\
  \upchord{G} So hoist up the John B’s sail\\
  See how the mainsail sets\\
  Call for the captain ashore let me go \upchord{D7} home\\
  Let me go \upchord{G} home \upchord{G7}\\
  I wanna go \upchord{C} home yeah \upchord{Am} yeah\\
  Well I \upchord{G} feel so broke up \upchord{D7} I wanna go \upchord{G} home\\
  \upchord{G} The first mate he got drunk\\
  And broke in the captain’s trunk\\
  The constable had to come and take him a\upchord{D7}way\\
  Sheriff John \upchord{G} Stone \upchord{G7}\\
  Why don’t you leave me a\upchord{C}lone yeah \upchord{Am} yeah\\
  Well I \upchord{G} feel so broke up \upchord{D7} I wanna go \upchord{G} home\\
  Chorus\\
  \upchord{G} The poor cook he caught the fits\\
  And threw away all my grits\\
  And then he took and he ate up all of my \upchord{D7} corn\\
  Let me go \upchord{G} home \upchord{G7}\\
  Why don’t they let me go \upchord{C} home \upchord{Am}\\
  This \upchord{G} is the worst trip \upchord{D7} I’ve ever been \upchord{G} on\\
  Chorus x 2

\section{The Times They Are a-Changin' / Bob Dylan}\label{sec:times_they_are_a_changin}
\Cmajor
\Dmajor
\Gmajor
\Aminor
\Eminor

Come \upchord{G}gather round \upchord{Em}people \upchord{C}wherever you \upchord{G}roam

And \upchord{G}admit that the \upchord{Em}waters \upchord{C}around you have \upchord{D}grown

And \upchord{G}accept it that \upchord{Em}soon you'll be \upchord{C}drenched to the \upchord{G}bone

If your \upchord{G}time to \upchord{Am}you is worth \upchord{D}savin'

So you \upchord{D}better start \upchord{C}swimming or you'll \upchord{G}sink like a \upchord{D}stone

For the \upchord{G}times, they \upchord{C}are a-\upchord{D}chang-\upchord{G}in'\\


Come \upchord{G}writers and \upchord{Em}critics who \upchord{C}prophesise with your \upchord{G}pen 

And \upchord{G}keep your eyes \upchord{Em}wide the chance \upchord{C}won't come \upchord{D}again

And \upchord{G}don't speak too \upchord{Em}soon for the wheel's \upchord{C}still in \upchord{G}spin

And there's \upchord{G}no tellin' \upchord{Am}who that it's \upchord{D}namin'

For the \upchord{D}loser \upchord{C}now will be \upchord{G}later to \upchord{D}win

For the \upchord{G}times they \upchord{C}are a-\upchord{D}chang\upchord{G}in'\\


Come \upchord{G}mothers and \upchord{Em}fathers \upchord{C}throughout the \upchord{G}land

And \upchord{G}don't criti\upchord{Em}cize what you \upchord{C}don't under \upchord{D}stand

Your \upchord{G}sons and your \upchord{Em}daughters are \upchord{C}beyond your \upchord{G}command

Your \upchord{G}old road is \upchord{Am}rapidly \upchord{D}agin'

Please \upchord{D}get out of the \upchord{C}new one if you \upchord{G}can't lend a \upchord{D}hand

For the \upchord{G}times they \upchord{C}are a- \upchord{D}chan\upchord{G}gin'\\


Come \upchord{G}senators, \upchord{Em}congressmen \upchord{C}please heed the \upchord{G}call

Don't \upchord{G}stand in the \upchord{Em}doorway, don't \upchord{C}block up the \upchord{D}hall

For \upchord{G}he that gets \upchord{Em}hurt will be \upchord{C}he who has \upchord{G}stalled

There's a \upchord{G}battle out \upchord{Am}side and it's \upchord{D}ragin'

It'll \upchord{D}soon shake your \upchord{C}windows and \upchord{G}rattle your \upchord{D}walls

For the \upchord{G}times they \upchord{C}are a- \upchord{D}chang\upchord{G}in'\\


The \upchord{G}line it is \upchord{Em}drawn the \upchord{C}curse it is \upchord{G}cast

The \upchord{G}slow one \upchord{Em}now will \upchord{C}later be \upchord{D}fast

As the \upchord{G}present \upchord{Em}now will \upchord{C}later be \upchord{G}past

The \upchord{G}order is \upchord{Am}rapidly \upchord{D}fadin'

And the \upchord{D}first one \upchord{C}now will \upchord{G}later be \upchord{D}last

For the \upchord{G}times they \upchord{C}are a- \upchord{D}chang\upchord{G}in'
\section{William Harker / Skinny Lister / George Thomas}\label{sec:william_harker}
  {\small chords}
  
  \Cmajor
  \Aminor
  \Dmajor
  
  Me \upchord{G} name is William Harker I'm a rake and a roarin' blade
  \upchord{C} Revelry is \upchord{G} my delight and \upchord{Am} roguery's me \upchord{D} trade
  If you \upchord{G} chuck a penny in the ole grey goose you're sure to hit a nail
  \upchord{C} There's no one that \upchord{G} I have done, no \upchord{Am} friendship \upchord{D} I've be\upchord{G}trayed
  CHORUS
  And it's \upchord{G} hold me hand me hearties, step with the heel and toe
  \upchord{C} Rolling down, \upchord{G} strolling down to the \upchord{Am} public house we'll \upchord{D} go
  We've a \upchord{G} flute and a fiddle and a derry down diddle, and a rattle on the ole
  banjo
  \upchord{C} Rolling down, \upchord{G} strolling down to the \upchord{Am} public \upchord{D} house we'll \upchord{G} go
  VERSE 2
  I've \upchord{G} kissed some girls in Belgium, I've cuddled some girls in Spain
  But the \upchord{C} girls I kiss on a \upchord{G} Saturday night are the \upchord{Am} girls I'd kiss \upchord{D}
  again
  When we \upchord{G} cut a caper on the bar room floor for neither lust nor gain
  If \upchord{C} we're not lovers when we're \upchord{G} homeward bound, it’s \upchord{Am} good friends
  \upchord{D} we'll \upchord{G} remain
  >>>Repeat CHORUS<<<
  VERSE 3
  I've \upchord{G} got a dog named Barker, he likes a drop of beer
  He \upchord{C} gets right drunk, rolls \upchord{G} on the floor and he \upchord{Am} grins from ear to \upchord{D}
  ear
  And by \upchord{G} day you'll see us walking over woodland, hill and vale
  \upchord{C} Every night we'll \upchord{G} both get tight on \upchord{Am} Mrs \upchord{D} Sturgen's \upchord{G} ale
  INSTRUMENTAL - \upchord{G} / \upchord{C} / \upchord{G} / \upchord{Am} / \upchord{D} / \upchord{G} / \upchord{C} / \upchord{G} / \upchord{Am} / \upchord{D} / \upchord{G}
  VERSE 4
  Now \upchord{G} some like a cool clear larger, others like a bottle of pale
  The \upchord{C} boys all titter at a \upchord{G} pint of bitter and some \upchord{Am} drink Anstell's \upchord{D}
  ale (God help them!)
  I'm \upchord{G} pegging out for a jet black stout, it's beauty to aspire
  With me \upchord{C} hand the bar I'll \upchord{G} swear Ooo-arr! I'll \upchord{Am} drink it \upchord{D} till I \upchord{G} die
  >>>Repeat CHORUS<<< (slow down at last line to end)

\section{Wuthering Heights / Kate Bush}\label{sec:wuthering_heights}
  {\small chords}
  
  \Fmajor
  \Emajor
  \CsharpMajor
  \AflatMajor
  \Dminor
  \Gmajor
  \Cmajor
  
  \upchord{A} Out on the wily, \upchord{F} windy moors, we’d \upchord{E} roll and
  fall in \upchord{C\#} green.//
  \upchord{A} You had a temper \upchord{F} like my jealou-\upchord{E}-sy, too hot,
  too \upchord{C\#} greedy.//
  \upchord{A} How could you leave me, \upchord{F} when I needed to \upchord{E}
  possess you? \upchord{C\#} I hated you. \upchord{Ab} I loved you, too.//
  \upchord{A few players strumming}//
  \upchord{F} Bad dreams in the \upchord{E} night. \upchord{F} You told me I was
  \upchord{E} going to lose the fight. \upchord{F} Leave behind my \upchord{E}
  Wuthering, Wuthering, Wuthering Heights.//
  \upchord{Everyone strumming}//
  Heath-\upchord{F}-cliff, \upchord{Dm} it’s \upchord{G} me, I’m Cathy, I’ve come
  \upchord{C} home. \upchord{F} I ́m so cold, let me \upchord{G} in-a-your \upchord{C} win\upchord{F}-dow.//
  Heath-\upchord{F}-cliff, \upchord{Dm} it’s \upchord{G} me, I’m Cathy, I’ve come
  \upchord{C} home. \upchord{F} I ́m so cold, let me \upchord{G} in-a-your \upchord{C} win\upchord{F}-dow.//
  \upchord{A} Ooh, it gets dark! \upchord{F} It gets lonely, \upchord{E} on the
  other \upchord{C\#} side from you.//
  \upchord{A} I pine a lot. \upchord{F} I find the lot \upchord{E} faster with \upchord{C\#}
  out you.//
  \upchord{A} I’m coming back, love, \upchord{F} cruel Heathcliff, \upchord{E} my
  one dream, \upchord{C\#} my only \upchord{Ab} master.//
  \upchord{F} Too long I \upchord{E} roamed in the night.//
  \upchord{F} I’m coming back to his \upchord{E} side, to put it right.//
  \upchord{F} I’m coming home to \upchord{E} Wuthering, Wuthering,
  Wuthering Heights.//
  Heath-\upchord{F}-cliff, \upchord{Dm} it’s \upchord{G} me, I’m Cathy, I’ve come
  \upchord{C} home. \upchord{F} I ́m so cold, let me \upchord{G} in-a-your \upchord{C} win\upchord{F} dow.//
  Heath-\upchord{F}-cliff, \upchord{Dm} it’s \upchord{G} me, I’m Cathy, I’ve come
  \upchord{C} home. \upchord{F} I ́m so cold, let me \upchord{G} in-a-your \upchord{C} win\upchord{F}-dow//
  \upchord{Am} Ooh! Let me \upchord{G} have it.//
  Let me \upchord{F} grab your \upchord{G} soul away.//
  \upchord{Am} Ooh! Let me \upchord{G} have it.//
  Let me \upchord{F} grab your \upchord{G} soul away.//
  \upchord{Am} You know it’s \upchord{G} me, Cath-\upchord{F}-y!//
  Heath-\upchord{F}-cliff, \upchord{Dm} it’s \upchord{G} me, I’m Cathy, I’ve come
  \upchord{C} home. \upchord{F} I ́m so cold, let me \upchord{G} in-a-your \upchord{C} win-
  \upchord{F}-dow.//
  Heath-\upchord{F}-cliff, \upchord{Dm} it’s \upchord{G} me, I’m Cathy, I’ve come
  \upchord{C} home. \upchord{F} I ́m so cold, let me \upchord{G} in-a-your \upchord{C} win-
  \upchord{F}-dow.//
  Heath-\upchord{F}-cliff, \upchord{Dm} it’s \upchord{G} me, I’m Cathy, I’ve come
  \upchord{C} home. \upchord{F} I ́m so cold...//
  \upchord{Single strum:} \upchord{F}//
  \upchord{Dm} \upchord{G} \upchord{C}, cha cha cha//

% \section{YMCA / Village People}\label{sec:ymca}
  {\small chords}
  
  \Aminor
  \Fmajor
  \Gseven
  
  \upchord{C} Young man there's no need to feel down\\
  I said \upchord{Am} young man pick yourself off the ground\\
  I said \upchord{F} young man cause you're in a new town\\
  There's no \upchord{G7} need to be unhappy\\
  \upchord{C} Young man there's a place you can go\\
  I said \upchord{Am} young man when you're short on your dough\\
  You can \upchord{F} stay there and I'm sure you will find\\
  Many \upchord{G7} ways to have a good time (1,2,3,4,5..)\\
  Chorus: It's fun to stay at the \upchord{C} YMCA it's fun to stay at the \upchord{Am} YMCA\\
  They have \upchord{F} everything for young men to enjoy\\
  You can \upchord{G7} hang out with all the boys\\
  It's fun to stay at the \upchord{C} YMCA it's fun to stay at the \upchord{Am} YMCA\\
  You can \upchord{F} get yourself cleaned you can have a good meal\\
  You can \upchord{G7} do whatever you feel\\
  \upchord{C} Young man are you listening to me\\
  I said \upchord{Am} young man what do you want to be\\
  I said \upchord{F} young man you can make real your dreams\\
  But you \upchord{G7} got to know this one thing\\
  \upchord{C} No man does it all by himself\\
  I said \upchord{Am} young man put your pride on the shelf\\
  And just \upchord{F} go there to the YMCA\\
  I'm \upchord{G7} sure they can help you today (1,2,3,4,5..)\\
  \upchord{C} Young man I was once in your shoes\\
  I said \upchord{Am} I was down and out with the blues\\
  I felt \upchord{F} no man cared if I were alive\\
  I felt \upchord{G7} the whole world was so tight\\
  That's when \upchord{C} someone came up to me\\
  And said \upchord{Am} young man take a walk up the street\\
  There's a \upchord{F} place there called the YMCA\\
  They can \upchord{G7} start you back on your way (1,2,3,4,5..)\\
  Chorus x 2

%% END SONGS NZ

\nolinenumbers
% chapter songs_n_z (end)

% \chapter{Appendix A: New Songs} % (fold)
% \label{prt:appendix_a}
% \minitoc
% \Large
% \linenumbers
% \modulolinenumbers[2]
%% START SONGS APPENDIX_A
%
%% END SONGS APPENDIX_A
% \nolinenumbers
% chapter appendix_a (end)

\end{document}
\end
